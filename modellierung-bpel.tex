\section{Web Services Business Process Execution Language}\label{mod-bpel}

\subsection{Hintergrund}
Die Web Services Business Process Execution Language (WS-BPEL) ist ein Standard, der ein Modell und die Grammatik für die Beschreibung von Geschäftsprozessen definiert, die auf der Interaktion von Prozessen und deren Partnern basieren (vgl. \citep{OASIS2007}, S.8).

Einzelne Prozesse werden hierbei als Web-Services (XXX) angeboten. Web-Services im Sinne dieses und anderer WS-* Standards basieren auf folgenden Technologien (vgl. \citep{OASIS2007}, S.7):
\begin{description}
\item[Simple Object Access Protocol (SOAP):] Definiert das XML Nachrichtenaustausch Protokoll für grundlegende Service Interoperabilität.
\item[Web Services Definition Language (WSDL):] Stellt eine grundlegende Grammatik zur Beschreibung von Services zur Verfügung.
\item[Universal Description, Discovery and Integration (UDDI):] Stellt die Infrastruktur zur Veröffentlichung und Suche der Services zur Verfügung.
\end{description}

WS-BPEL ergänzt diese Standards um Sprachkonstrukte, mit denen die Interaktion zwischen Webservices definiert werden kann und Geschäftsprozessen erstellt werden können. WS-BPEL basiert auf XML und kann mit anderen Standards aus der WS-* Familie (bsp. WS-Policy) kombiniert werden. WS-BPEL operiert in einem SOA (Service Oriented Architecture) Umfeld (vgl. \citep{BPEL1}, S.313-315).

Die WS-BPEL Spezifikation entstand durch die Kombination der Standards Web Service Flow Language (WSFL) von IBM und XLANG von Microsoft. Beide hatten die Komposition von Services zum Ziel, aber mit unterschiedlichen Ansätzen (vgl. \citep{BPEL1}, S.316). Gemeinsam mit BEA wurde BPEL4WS (Business Process Execution Language for Web Services) und die Standards WS-Transaction und WS-Coordination im Juli 2002 veröffentlicht. Im Mai 2003 wurde die Version 1.1 unter den neuen Namen WS-BPEL bei der Organisation OASIS\footnote{OASIS: Organization for the Advancement of Structured Information Standards, \url{http://www.oasis-open.org/}.} zur Standardisierung eingereicht (vgl. \citep{BPEL2}, S.101). Der aktuelle Standard trägt die Versionsnummer 2.0 und ist im April 2007 von OASIS veröffentlicht worden (vgl. \citep{OASIS2007}).

\subsection{Die Struktur von WS-BPEL Prozessen}









% END OF DOCUMENT
