\section{Diskussion der vorgestellten Modellierungssprachen}\label{mod-sum}

Petrinetze können als Grundlage der meisten Prozessmodellierungssprachen angesehen werden, wie zum Beispiel die zuvor vorgestellten Aktivitätsdiagramme und BPMN. Petrinetze haben das Konzept nebenläufiger Prozesse eingeführt, die durch Tokens koordiniert werden. Sie besitzen Möglichkeiten zur Aufteilung, Synchronisation, Zusammenführung und der Wahl alternativer Pfade. Die ursprüngliche Form unterstützt aber keine typisierten Tokens. Es gibt aber zahlreiche Erweiterungen zu den Petrinetzen, die sie für bestimmte andere Anwendungsgebiete optimieren.

Aktivitätsdiagramme seit der UML Version 2 sind an Petrinetze angelehnt. Grundlegende Sprachkonstrukte wie Decision, Fork, Join und Merge lassen sich auch mit Petrinetzen umsetzen. Aktivitätsdiagramme haben aber eine große Zahl an Erweiterungen und semantischen Besonderheiten, die in dieser Form in Petrinetzen nicht existieren. Die Arbeiten von Störrle und Hausmann (Towards a Formal Semantics of UML 2.0 Activities) sowie von Schattkowsky und Förster (On the Pitfalls of UML 2 Activity Modeling) arbeiten diese Unterschiede heraus (vgl. \citep{StorrleH2005} und \citep{SForster2007}). Die größere Vielfalt an Elementen in Aktivitätsdiagrammen ist einerseits den Anforderungen der Geschäftsprozessmodellierung, andererseits den Anforderungen von Software- und Hardwaredesign geschuldet \citep{SForster2007}. Die Einbettung in den gesamten UML Standard und der Anspruch eines universellen Werkzeugs zur Verhaltensmodellierung von Systemen ist eine große Stärke der Aktivitätsdiagramme. Dies bedingt aber eine große Komplexität, die eine Umsetzung des Standards in Werkzeugen erschwert. Deshalb wurden Compliance Levels von der OMG definiert, die eine Teilweise Umsetzung des Standards ermöglichen und trotzdem Kompatibilität auf dem entsprechenden Compliance Level garantieren (siehe \rref{mod-uml-compliance}).

Die Konzepte der BPMN sind denen von Aktivitätsdiagrammen sehr ähnlich. Stephen A. White hat in der Arbeit "Process Modeling Notations and Workflow Patterns" 21 häufig in Geschäftsprozessen auftretende Patterns analysiert und ihre Umsetzbarkeit mit BPMN und mit Aktivitätsdiagrammen untersucht ((vgl. \citep{White20042}). Es konnte dabei nur ein Pattern identifiziert werden, für den es in Aktivitätsdiagrammen keine direkte Entsprechung gab, der aber durch Kombination anderer Elemente ebenso umgesetzt werden konnte. Da sich die BPMN aber nur auf das Feld der Geschäftsprozessmodellierung stützt, können viele Prozesse klarer verständlich und weniger ausführlich modelliert werden.
Beide Modellierungssprachen unterstützen das Abbilden von Daten, die im Prozess verwendet werden, wobei in der BPMN diese Artefakte aber nicht den Prozessfluss steuern. Die BPMN Spezifikation sieht vor, dass Artefakte von Toolherstellern oder Anwendern erweitert werden können. In UML kann das Metamodell durch Stereotypen erweitert werden, für die auch eine graphische Darstellung durch Bilder definiert werden kann. Beide Standards unterstützen die Modellierung von Ressourcen und Zuständigkeiten durch Partitionen (UML) beziehungsweise Pool und Swimlanes (BPMN).
Die Spezifikation der BPMN basiert im Gegensatz zu den Aktivitätsdiagrammen nicht auf einen formalisierten Metamodell und hat auch keinen Standard zur Serialisierung und zum Austausch von Modellen definiert. Dieser Umstand schafft Kompatibilitätsprobleme beim Austausch von Modellen zwischen Werkzeugen, die es mit UML basierten Werkzeugen nicht geben sollte.
% XXX beispiel und begründung warum der austausch von modellen mit uml basierten werkzeugen ebenfalls schwierig ist...
Diese Probleme wurden erkannt und werden voraussichtlich mit dem kommenden Standard der BPMN, Version 2, gelöst (vgl. \citep{BPMN2007}).

Die AMREP basiert auf dem Standard für UML Aktivitätsdiagramme, da die Interoperabilität mit anderen Diagrammarten der UML und mit UML kompatiblen Werkzeugen, das mit der MOF kompatible Metamodell der UML und die breite Anwendbarkeit für eine Vielzahl an Einsatzszenarien als wesentliche Vorteile gelten. Dennoch lag die Implementierung des gesamten Standards für Aktivitätsdiagramme außerhalb der Möglichkeiten dieser Diplomarbeit. Es wurde ein Metamodell mit einem Subset an Elementen aus dem UML Metamodell definiert, wobei für einige Elemente auch eine andere Semantik definiert wurde und Vereinfachungen getroffen wurden. Deshalb ist das AMREP Metamodell nicht vollständig kompatibel zum UML Metamodell. Welche dieser Vereinfachungen vorgenommen wurden und warum diese Entscheidungen getroffen wurden, wird in den nächsten Kapiteln erläutert.

% END OF DOCUMENT
