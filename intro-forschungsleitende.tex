\chapter{Forschungsleitende Ansätze}


In den vorhergehenden Kapiteln wurden verschiedene Ansätze der Geschäftsprozessmodellierung und Modellausführung vorgestellt. Folgende Ansätze dienen als Grundlage für diese Diplomarbeit:

\begin{enumerate}
\item Geschäftsprozesse lassen sich mit UML2 Aktivitätsdiagrammen modellieren (vgl. \citep{White2004}).
\item Modelle, die auf einer formalen Modellierungssprache basieren, können ausgeführt und interpretiert werden (vgl. \citep{OMG2008}, S.12).
\item Die Ausführung kann direkt durch Interpretation des Modells erfolgen (vgl. \citep{Crane2009}, S. 180ff.).
\end{enumerate}

Des Weiteren soll diese Diplomarbeit folgende Annahme beweisen:
\begin{itemize}
\item Von Maschinen Interpretierte Modelle können zur Laufzeit geändert werden.
\end{itemize}


% END OF DOCUMENT
