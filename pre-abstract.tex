%problem field
%purpose and methodology
%results and conclusion

\section*{Abstract (English)}
Business process modeling techniques have been helpful to understand the activities which are done in enterprises. However, traditional information systems for enterprises show little flexibility in adapting business processes found in real environments. The BPEL standard tries to meet these shortcomings but is limited to web services. A more general modeling approach can be found in the standards published by the Object Management Group, specifically the UML standard.

The Activity Diagram specification of the UML standard can be used to model behavioral systems, including business processes. Formal models with sufficient semantics can be executed. The semantics is ensured by a well defined meta model which provides a common basis for executable models. Furthermore, executable models can be interpreted rather than compiled to a target Language. A model interpreter also allows run-time changing of the executed behavior.

In this diploma thesis a prototype was built in the programing language python, which consists of a customized metamodel and a runtime interpreter for activity models. Such a activity model runtime engine does not yet exist for Python. The functionality was proven for a use case by unit tests. A restriction is the limited compatibility to the UML standard for activity diagrams. Nevertheless, the results show that modeled processes can be successfully used with the Activity Model Runtime Engine for Python. More use cases have to be implemented to prove the application of the customized metamodel and token semantics for the respective purpose.

\newpage

\section*{Abstract (Deutsch)}
Geschäftsprozess-Modellierungstechniken haben das Verständnis von Ab\-läu\-fen in Unternehmen erhöht. Traditionelle Informationssysteme bieten allerdings wenig Flexibilität, Abläufe, die im realen Umfeld verrichtet werden, zu implementieren. Der Standard BPEL versucht diese Lücke zu schließen, ist aber auf Web-Services beschränkt. Ein generellerer Modellierungsansatz kann in den von der Object Management Group veröffentlichten Standards, speziell dem UML-Standard, gefunden werden.

Die Aktivitätsdiagramm-Spezifikation des UML-Standards kann zur Modellierung des Verhaltens von Systemen inklusive der Modellierung von Geschäfts\-prozessen verwendet werden. Formale Modelle mit ausreichend semantischer Information können ausgeführt werden. Die Modellsemantik wird durch ein Metamodell sichergestellt, das als Basis für ausführbare Modelle dient. Darüber hinaus können solche Modelle interpretiert werden, anstatt sie in eine Zielsprache zu kompilieren. Ein Modellinterpreter ermöglicht auch das Ändern von Modellen zur Laufzeit.

In dieser Diplomarbeit wurde eine prototypische Implementierung eines angepassten Metamodells und eines Modellinterpreters für Aktivitäten in der Programmiersprache Python erstellt. Eine solche Activity Model Runtime Engine existiert für diese Programmiersprache noch nicht. Die Funktionalität wurde anhand eines Use Case durch Unit-Tests bewiesen. Eine Einschränkung ist die limitierte Kompatibilität zum UML Standard für Aktivitätsdiagramme. Das Ergebnis zeigt aber, dass  modellierte Abläufe mit der Activity Model Runtime Engine für Python erfolgreich umgesetzt werden können. Weitere Use Cases müssen noch implementiert werden, um die Anwendbarkeit des angepassten Metamodells und der angepassten Token-Semantik für den jeweiligen Zweck zu prüfen.


% Zu diesem Zweck wurde ein Metamodell definiert, das Konzepte des UML2 Metamodells für Aktivitäten beinhaltet und als Basis für interpretierbare Modelle dient.


% END OF DOCUMENT
